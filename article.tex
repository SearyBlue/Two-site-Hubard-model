\documentclass[12pt]{article}
\usepackage{common}
\usepackage[margin=1in]{geometry}
\setlength{\parindent}{0in}
\newcommand{\C}[2]{\ensuremath{c_{{#1}{#2}}^\dagger}}
\renewcommand{\c}[2]{\ensuremath{c_{{#1}{#2}}}}
\begin{document}
\begin{titlepage}

\begin{center}
\bf{\Large{.......................}}\\
\vspace{10mm}
\bf{\Large{.......................} } \\
\vspace{5mm}
\rr{\it{\large{.......................}}} \\
\vspace{15mm}

\bf{\Large{.......................}}
\end{center}

\end{titlepage}

\section{Exact diagonalization of the two-site Hubbard model}

\pp[The Hamiltonian]
\begin{zz}
\ham = -t\sum_\sigma\rr{\C{1}{\sigma}\c{2}{\sigma}+\C{2}{\sigma}\c{1}{\sigma}} + U\sum_i\hat{n}_{i\uparrow}\hat{n}_{i\downarrow} -\mu \hat{N}
\end{zz}

I have two lattice sites, indexed by 1 and 2, occupied by electrons. \(\mu\) is the chemical potential, \C{i}{\sigma} and \(c_{i\sigma}\) are the fermionic creation and annihilation operators at the i\uu{th} site, with spin-index \(\sigma\). \(\sigma\) can take values \(\uparrow\) and \(\downarrow\), denoting spin-up and spin-down states respectively. \(\hat{n}_{i\sigma}=\C{i}{\sigma} \c{i}{\sigma}\) is the number operator for the \(i^{th}\) site and at spin-index \(\sigma\); it counts the number of fermions with the designated quantum numbers. \(\hat{N}= \sum_{i\sigma}\hat{n}_{i\sigma}\) is the total number operator; it counts the total number of fermions at all sites and spin-indices. \it t is the hopping strength; the more the t, the more are the electrons likely to hop between sites. \it U is the on-site repulsion cost; it represents the increase in energy when two electrons occupy the same site.

\subsection{Symmetries of the problem}
The following operators commute with the Hamiltonian.
\begin{enumerate}
\item\bf{Total number operator}:
\begin{zz}
\qq{\ham, \hat N}=0
\end{zz}
The proof is as follows: The last term in the Hamiltonian is the number operator itself. Ignoring that, there are three terms that I need to individually consider.

\begin{itemize}
\item \(\C{1}{\sigma}\c{2}{\sigma}\)\\

\begin{zz}
\qq{\C{1}{\sigma}\c{2}{\sigma}, \hat{N}} &= \sum_{i\sigma^\prime}\qq{\C{1}{\sigma}\c{2}{\sigma},\hat{n}_{i\sigma^\prime}} \\ 
&= \sum_{i\sigma^\prime}\qq{\C{1}{\sigma}\c{2}{\sigma},\C{i}{\sigma^\prime}\c{i}{\sigma^\prime}} \\
&= \sum_{i\sigma^\prime}\rr{\C{1}{\sigma} \qq{\c{2}{\sigma},\C{i}{\sigma^\prime}\c{i}{\sigma^\prime}}+\qq{\C{1}{\sigma},\C{i}{\sigma^\prime}\c{i}{\sigma^\prime}}\c{2}{\sigma}}
\end{zz}
Because the electrons on different sites are distinguishable, creation and anhillation operators of different sites will commute among themselves.
\begin{zz}
\qq{\C{1}{\sigma}\c{2}{\sigma}, \hat{N}} &= \sum_{i\sigma^\prime}\rr{\C{1}{\sigma} \delta_{i,2}\qq{\c{2}{\sigma},\C{2}{\sigma^\prime}\c{2}{\sigma^\prime}}+\delta_{i,1}\qq{\C{1}{\sigma},\C{1}{\sigma^\prime}\c{1}{\sigma^\prime}}\c{2}{\sigma}} \\
&= \sum_{\sigma^\prime}\rr{\C{1}{\sigma} \cc{\c{2}{\sigma},\C{2}{\sigma^\prime}}\c{2}{\sigma^\prime}-\C{1}{\sigma^\prime}\cc{\c{1}{\sigma^\prime},\C{1}{\sigma}}\c{2}{\sigma}} \\
&= \sum_{\sigma^\prime}\rr{\C{1}{\sigma} \delta_{\sigma,\sigma^\prime} \c{2}{\sigma^\prime}-\C{1}{\sigma^\prime} \delta_{\sigma,\sigma^\prime} \c{2}{\sigma}} = 0
\end{zz}

\item \(\C{2}{\sigma}\c{1}{\sigma}\): Since the operator \(\hat{N}\) is symmetric with respect to the site indices 1 and 2, we can go through the last proof again with the site indices 1 and 2 exchanged and since the proof does not depend on the site indices, this commutator will also be zero.

\item \(\hat{n}_{i\uparrow}\hat{n}_{i\downarrow}\):
\begin{zz}
\qq{\hat{n}_{i\uparrow}\hat{n}_{i\downarrow}, \hat{N}} &= \sum_{j\sigma}\qq{\hat{n}_{i\uparrow}\hat{n}_{j\downarrow},\hat{n}_{j\sigma}} \\
&= \sum_{\sigma}\qq{\hat{n}_{i\uparrow}\hat{n}_{i\downarrow},\hat{n}_{i\sigma}} \\
&= \hat{n}_{i\uparrow}\qq{\hat{n}_{i\downarrow},\hat{n}_{i\uparrow}}+\qq{\hat{n}_{i\uparrow},\hat{n}_{i\downarrow}}\hat{n}_{i\downarrow} = 0
\end{zz}

\end{itemize}

\item 

\end{enumerate}
\end{document}
