\documentclass{article}
\usepackage{common}
\usepackage[margin=1in]{geometry}
\renewcommand\arraystretch{2.5}
\begin{document}
\begin{titlepage}

\begin{center}
\bf{\Large{.......................}}\\
\vspace{10mm}
\bf{\Large{.......................} } \\
\vspace{5mm}
\rr{\it{\large{.......................}}} \\
\vspace{15mm}

\bf{\Large{.......................}}
\end{center}

\end{titlepage}

\section{Exact diagonalization of the two-site Hubbard model}

\pp[The Hamiltonian]
\beq
\ham = -t\sum_\sigma\rr{\C{1}{\sigma}\c{2}{\sigma}+\C{2}{\sigma}\c{1}{\sigma}} + U\sum_i\hat{n}_{i\uparrow}\hat{n}_{i\downarrow} -\mu \hat{N}
\eeq

I have two lattice sites, indexed by 1 and 2, occupied by electrons. \(\mu\) is the chemical potential, \C{i}{\sigma} and \(c_{i\sigma}\) are the fermionic creation and annihilation operators at the i\uu{th} site, with spin-index \(\sigma\). \(\sigma\) can take values \(\uparrow\) and \(\downarrow\), denoting spin-up and spin-down states respectively. \(\hat{n}_{i\sigma}=\C{i}{\sigma} \c{i}{\sigma}\) is the number operator for the \(i^{th}\) site and at spin-index \(\sigma\); it counts the number of fermions with the designated quantum numbers. \(\hat{N}= \sum_{i\sigma}\hat{n}_{i\sigma}\) is the total number operator; it counts the total number of fermions at all sites and spin-indices. \it t is the hopping strength; the more the t, the more are the electrons likely to hop between sites. \it U is the on-site repulsion cost; it represents the increase in energy when two electrons occupy the same site.

\subsection{Symmetries of the problem}
The following operators commute with the Hamiltonian.
\begin{enumerate}
\item\bf{Total number operator}: \(\qq{\ham, \hat N}=0\).
\begin{proof}
The last term in the Hamiltonian is the number operator itself. Ignoring that, there are three terms that I need to individually consider.
\begin{itemize}
\item \(\C{1}{\sigma}\c{2}{\sigma}\)
\beq[commutator]
\qq{\C{1}{\sigma}\c{2}{\sigma},\hat{n}_{i\sigma^\prime}} &= \qq{\C{1}{\sigma}\c{2}{\sigma},\C{i}{\sigma^\prime}\c{i}{\sigma^\prime}} \\
&= \C{1}{\sigma} \qq{\c{2}{\sigma},\C{i}{\sigma^\prime}\c{i}{\sigma^\prime}}+\qq{\C{1}{\sigma},\C{i}{\sigma^\prime}\c{i}{\sigma^\prime}}\c{2}{\sigma} \\
&= \delta_{i,2}\:\C{1}{\sigma} \qq{\c{2}{\sigma},\C{2}{\sigma^\prime}\c{2}{\sigma^\prime}}+\delta_{i,1}\qq{\C{1}{\sigma},\C{1}{\sigma^\prime}\c{1}{\sigma^\prime}}\c{2}{\sigma} \\
&= \delta_{i,2}\:\C{1}{\sigma} \cc{\c{2}{\sigma},\C{2}{\sigma^\prime}}\c{2}{\sigma^\prime} - \delta_{i,1}\C{1}{\sigma^\prime}\cc{\c{1}{\sigma^\prime},\C{1}{\sigma}}\c{2}{\sigma} \\
&= \delta_{\sigma,\sigma^\prime}\C{1}{\sigma}\c{1}{\sigma}\rr{\delta_{i,2} - \delta_{i,1}}
\eeq
The third line follows because the electrons on different sites are distinguishable and hence, the \it{creation and anhillation operators of different sites will commute among themselves}. Therefore,
\beq
\qq{\C{1}{\sigma}\c{2}{\sigma}, \hat{N}} &= \sum_{i\sigma^\prime}\qq{\C{1}{\sigma}\c{2}{\sigma},\hat{n}_{i\sigma^\prime}} = \C{1}{\sigma}\c{1}{\sigma}\sum_{i=\{1,2\}}\rr{\delta_{i,2} - \delta_{i,1}} = 0
\eeq

\item \(\C{2}{\sigma}\c{1}{\sigma}\): Since the operator \(\hat{N}\) is symmetric with respect to the site indices 1 and 2, I can go through the last proof again with the site indices 1 and 2 exchanged and since the proof does not depend on the site indices, this commutator will also be zero.

\item \(\hat{n}_{i\uparrow}\hat{n}_{i\downarrow}\):
\beq[commutator_2]
\qq{\hat{n}_{i\uparrow}\hat{n}_{j\downarrow},\hat{n}_{j\sigma}} &= \hat{n}_{i\uparrow}\qq{\hat{n}_{i\downarrow},\hat{n}_{j\sigma}} - \qq{\hat{n}_{i\uparrow},\hat{n}_{j\sigma}}\hat{n}_{i\downarrow} \\
&= \delta_{ij}\rr{\hat{n}_{i\uparrow}\qq{\hat{n}_{i\downarrow},\hat{n}_{i\sigma}} - \qq{\hat{n}_{i\uparrow},\hat{n}_{i\sigma}}\hat{n}_{i\downarrow}} \\
&= \delta_{ij}\rr{\delta_{\sigma\uparrow}\hat{n}_{i\uparrow}\qq{\hat{n}_{i\downarrow},\hat{n}_{i\uparrow}} - \delta_{\sigma\downarrow}\qq{\hat{n}_{i\uparrow},\hat{n}_{i\downarrow}}\hat{n}_{i\downarrow}} \\
&= \delta_{ij} \rr{\delta_{\sigma\downarrow}\hat{n}_{i\downarrow} - \delta_{\sigma\uparrow}\hat{n}_{i\uparrow}} \qq{\hat{n}_{i\uparrow},\hat{n}_{i\downarrow}} \\
&= \delta_{ij} \rr{\delta_{\sigma\downarrow}\hat{n}_{i\downarrow} - \delta_{\sigma\uparrow}\hat{n}_{i\uparrow}} \rr{\C{i}{\uparrow}\c{i}{\uparrow}\C{i}{\downarrow}\c{i}{\downarrow}-\C{i}{\downarrow}\c{i}{\downarrow}\C{i}{\uparrow}\c{i}{\uparrow}}\\
&= \delta_{ij} \rr{\delta_{\sigma\downarrow}\hat{n}_{i\downarrow} - \delta_{\sigma\uparrow}\hat{n}_{i\uparrow}} \rr{\C{i}{\downarrow}\c{i}{\downarrow}\C{i}{\uparrow}\c{i}{\uparrow}-\C{i}{\downarrow}\c{i}{\downarrow}\C{i}{\uparrow}\c{i}{\uparrow}} = 0
\eeq
Therefore, \(\qq{\hat{n}_{i\uparrow}\hat{n}_{j\downarrow},\hat{N}} = \sum_{j,\sigma} \qq{\hat{n}_{i\uparrow}\hat{n}_{j\downarrow},\hat{n}_{j\sigma}} = 0\)
% \beq
% \qq{\hat{n}_{i\uparrow}\hat{n}_{i\downarrow}, \hat{N}} &= \sum_{j\sigma}\qq{\hat{n}_{i\uparrow}\hat{n}_{j\downarrow},\hat{n}_{j\sigma}} \\
% &= \sum_{\sigma}\qq{\hat{n}_{i\uparrow}\hat{n}_{i\downarrow},\hat{n}_{i\sigma}} \\
% &= \hat{n}_{i\uparrow}\qq{\hat{n}_{i\downarrow},\hat{n}_{i\uparrow}}+\qq{\hat{n}_{i\uparrow},\hat{n}_{i\downarrow}}\hat{n}_{i\downarrow} = 0
% \eeq

\end{itemize}
The total Hamiltonian is just a sum of the three terms; since the number operator commutes individually with these terms, it obviously commutes with the total Hamiltonian.
\end{proof}
\item \bf{Magnetization operator}: \(\hat{S}^z_{tot} \equiv \frac{1}{2}\sum_i\rr{\hat{n}_{i\uparrow}-\hat{n}_{i\downarrow}}\), \(\qq{\ham, \hat{S}^z_{tot}}=0\).
\begin{proof}
The magnetization operator can be rewritten as \(\hat{S}^z_{tot} = \frac{1}{2}\sum_i\rr{\hat{n}_{i\uparrow}+\hat{n}_{i\downarrow}-2\hat{n}_{i\downarrow}} = \hat{N} - 2\sum_i\hat{n}_{i\downarrow}\). Since \(\hat{N}\) commutes with \(\ham\), I just need to prove that \(\qq{\ham, \sum_i\hat{n}_{i\downarrow}}\). From eq. \ref{commutator}, 
\beq
\qq{\C{1}{\sigma}\c{2}{\sigma},\sum_i\hat{n}_{i\downarrow}} &= \C{1}{\downarrow}\c{1}{\downarrow}\sum_{i=\{1,2\}}\rr{\delta_{i,2} - \delta_{i,1}} = 0
\eeq
Again using the symmetry of the magnetization operator with the exchange of indices, its obvious that
\qq{\C{2}{\sigma}\c{1}{\sigma},\sum_i\hat{n}_{i\downarrow}} = 0

Using eq. \ref{commutator_2}, \(\qq{\hat{n}_{i\uparrow}\hat{n}_{i\downarrow}, \hat{n}_{i\downarrow}} = 0\).

Finally, \(\qq{N, \hat{n}_{i\downarrow}}=\sum_{j\sigma}\qq{\hat{n}_{j\sigma},\hat{n}_{i\downarrow}} = \qq{\hat{n}_{i\uparrow},\hat{n}_{i\downarrow}} = \C{i}{\uparrow}\c{i}{\uparrow}\C{i}{\downarrow}\c{i}{\downarrow}-\C{i}{\downarrow}\c{i}{\downarrow}\C{i}{\uparrow}\c{i}{\uparrow} = 0\). Since \(\hat{S}^z_{tot}\) commutes with each part individually, it commutes with the total Hamiltonian.
\end{proof}

\item \bf{Two-site parity operator \(\hat{P}\)}: The action of \(\hat{P}\) is defined as follows. If \(\ket{\Psi_{\alpha\beta}}\) is a wavefunction with site indices \(\alpha\) and \(\beta\), 
\beq
\hat{P}\ket{\Psi(\alpha,\beta)} = \ket{\Psi(\beta,\alpha)}
\eeq
That is, it operates on each electron and reverses it's site indices. 
\begin{proof}
I now rewrite the Hamiltonian by explcitly showing the two site indices:
\beq
\ham(\alpha,\beta) = -t\sum_\sigma(\C{\alpha}{\sigma}\c{\beta}{\sigma}+\C{\beta}{\sigma}\c{\alpha}{\sigma}) + U(n_{\alpha\uparrow}n_{\alpha\downarrow}+n_{\beta\uparrow}n_{\beta\downarrow}) - \mu\sum_\sigma(n_{\alpha\sigma}+n_{\beta\sigma})
\eeq 
Its obvious that \(\ham\) is symmetric in the site indices: \(\ham(\alpha,\beta) = \ham(\beta,\alpha)\). This means that the eigenvalues also have this symmetry. Let \(\ket{\Phi(\alpha,\beta)}\) be an eigenstate of \(\ham(\alpha,\beta)\) with eigenvalue \(E(\alpha,\beta)\). Then,
\beq
\hat{P}\ham\ket{\Phi(\alpha,\beta)} &= E(\alpha,\beta) \hat{P}\ket{\Phi(\alpha,\beta)} = E(\beta,\alpha) \ket{\Phi(\beta,\alpha)} \\
&= \ham \ket{\Phi(\beta,\alpha)} = \ham \hat{P} \ket{\Phi(\alpha,\beta)} \\
\implies \ham\hat{P}\ket{\Phi(\alpha,\beta)} &= \hat{P}\ham\ket{\Phi(\alpha,\beta)}
\eeq
Since any general wavefunction can be expanded in terms of these wavefunctions and since both the operator are linear, the above result will also hold for a general wavefunction \(\ket{\Psi(\alpha,\beta)}\):
\beq
\ham \hat{P} \ket{\Psi(\alpha,\beta)} = \hat{P} \ham \ket{\Psi(\alpha,\beta)}
\implies [\ham, \hat{P}] = 0
\eeq
\end{proof}
\end{enumerate}
\subsection{Partitioning the Hilbert space}
THe Hamiltonian commutes with the three operators. This means that is possible to simultaneously diagonalize these four operators: \(\ham, \hat{N}, S_z^{tot}, \hat{P}\). I will be able to label the eigenstates of the total Hamiltonian using the eigenvalues of these operatos. First take the total number operator. \(\hat{N}\) can take four values for a two-site system, 1 through 4. The eigenstates labelled by a particular number, say N=2 will be orthogonal to the eigenstates labelled by another number, say N=4. This means each eigenvalue of \(\hat{N}\) will have a distinct subspace orthogonal to the other values of \(\hat{N}\). I will be able to diagonalize each such subspace independently of each other, because they will not have any overlap. This feature enables us to block-diagonalize the total Hamiltonian into four blocks, each block belonging to each value \(\hat{N}\). 

Inside each block, I will be able to repeat the procedure by next using the eigenvalues of \(S_z^{tot}\). Each block of the Hamiltonian will again break up to smaller blocks for each value of the total magnetization. The eigenvalues of parity operator provide a further partitioning of the blocks of magnetization.

From this point, all the states I will work with will necessarily be eigenfunctions of \(\hat{N}\), so it doesn't make sense to keep the last term in the Hamiltonian, \(\mu \hat{N}\). I redefine the Hamiltonian by absorbing this term: \(\ham \rightarrow \ham + \mu \hat{N} = -t\sum_\sigma\rr{\C{1}{\sigma}\c{2}{\sigma}+\C{2}{\sigma}\c{1}{\sigma}} + U\sum_i\hat{n}_{i\uparrow}\hat{n}_{i\downarrow}\). This will keep the eigenvectors unaltered, but will increase the eigenvalues by \(\mu N\), where N is the number of particles in the eigenstate I are considering.

\subsection{N = 1}
For writing the state kets, I use the following notation: \(\ket{\uparrow,\downarrow}\) means electron on site 1 has spin up and that on site 2 has spin-down. \(\ket{\downarrow, 0}\) means electron on site 1 has spin-down and there is no electron on site 2. 
\pp For one electron on two lattice sites, I start by writing down the eigenstates of \(S_z^{tot}\). For odd number of electrons, zero magnetization is not possible. So,
\begin{itemize}
\item \(S_z^{tot} = -1\): \(\ket{\downarrow,0}, \ket{0, \downarrow}\)
\item \(S_z^{tot} = +1\): \(\ket{\uparrow,0}, \ket{0, \uparrow}\)
\end{itemize}
Each eigenvalue will have a separate subspace and can be separately diagonalized. I need to find the matrix elements of \(\ham\) in these eigenkets. Since there is no possibility of two electrons occupying same site, I ignore the \(U\)-term for the time being. 
\subsubsection{\(S_z^{tot} = -1\)}
Let us first see the action of the Hamiltonian on the eigenfunctions with \(S_z^{tot} = -1\).
\beq
\ham\ket{\downarrow,0} = -t \C{2}{\downarrow}\c{1}{\downarrow}\ket{\downarrow,0} = -t\ket{0,\downarrow} \\
\ham\ket{0,\downarrow} = -t \C{1}{\downarrow}\c{2}{\downarrow}\ket{0,\downarrow} = -t\ket{\downarrow,0} \\
\eeq
We get the following matrix for this tiny subspace of the Hamiltonian:
\beq
\bordermatrix{~ & \ket{\downarrow,0} & \ket{0,\downarrow} \cr
              \ket{\downarrow,0} & 0 & -t \cr \\
              \ket{0,\downarrow} & -t & 0 \cr}
\eeq

The eigenvalues and eigenvectors of this matrix are \(\frac{\ket{\downarrow,0} \pm \ket{0,\downarrow}}{\sqrt{2}}\), with eigenvalues \(\mp t\). These are also the eigenvalues of the parity operator, as expected.
\beq
\hat{P}\rr{\ket{\downarrow,0} + \ket{0,\downarrow}} = \ket{0,\downarrow} + \ket{\downarrow,0} \implies \hat{P} = 1 \\
\hat{P}\rr{\ket{\downarrow,0} - \ket{0,\downarrow}} = \ket{0,\downarrow} - \ket{\downarrow,0} \implies \hat{P} = -1
\eeq
\subsubsection{\(S_z^{tot} = +1\)}
Now I look at the spin-up states.
\beq
\ham\ket{\uparrow,0} = -t \C{2}{\uparrow}\c{1}{\uparrow}\ket{\uparrow,0} = -t\ket{0,\uparrow} \\
\ham\ket{0,\uparrow} = -t \C{1}{\uparrow}\c{2}{\uparrow}\ket{0,\uparrow} = -t\ket{\uparrow,0} \\
\eeq
Clearly, this gives the same matrix as the spin-down states:
\beq
\bordermatrix{~ & \ket{\uparrow,0} & \ket{0,\uparrow} \cr
              \ket{\uparrow,0} & 0 & -t \cr \\
              \ket{0,\uparrow} & -t & 0 \cr}
\eeq
and hence similar eigenfunctions: \(\frac{\ket{\uparrow,0} \pm \ket{0,\uparrow}}{\sqrt{2}}\), with eigenvalues \(\mp t\).
\subsection{N=3}
I once again write down the eigenstates of \(S_z^{tot}\), this time with three electrons.
\begin{itemize}
\item \(S_z^{tot} = -1\): \(\ket{\uparrow\downarrow,\downarrow}, \ket{\downarrow,\uparrow\downarrow}\)
\item \(S_z^{tot} = +1\): \(\ket{\uparrow\downarrow,\uparrow}, \ket{\uparrow,\uparrow\downarrow}\)
\subsubsection{\(S_z^{tot} = -1\)}
\beq
\ham \ket{\uparrow\downarrow,\downarrow} = -t \C{2}{\uparrow}\c{1}{\uparrow}\ket{\uparrow\downarrow,\downarrow} + U\ket{\uparrow\downarrow,\downarrow} = -t\ket{\downarrow,\uparrow\downarrow} + U\ket{\uparrow\downarrow,\downarrow} \\
\ham \ket{\downarrow,\uparrow\downarrow} = -t \C{1}{\uparrow}\c{2}{\uparrow}\ket{\downarrow,\uparrow\downarrow} + U\ket{\downarrow,\uparrow\downarrow} = -t\ket{\uparrow\downarrow,\downarrow} + U\ket{\downarrow,\uparrow\downarrow}
\eeq
\beq
\bordermatrix{~ & \ket{\uparrow\downarrow,\downarrow} & \ket{\downarrow,\uparrow\downarrow} \cr
              \ket{\uparrow\downarrow,\downarrow} & U & -t \cr \\
              \ket{\downarrow,\uparrow\downarrow} & -t & U \cr}
\eeq
This matrix has eigenvalues \(U \mp t\), and corresponding eigenvectors \(\frac{\ket{\uparrow\downarrow,\downarrow} \pm \ket{\downarrow,\uparrow\downarrow}}{\sqrt{2}}\)

\subsubsection{\(S_z^{tot} = +1\)}
\beq
\ham \ket{\uparrow\downarrow,\uparrow} = -t \C{2}{\downarrow}\c{1}{\downarrow}\ket{\uparrow\downarrow,\uparrow} + U\ket{\uparrow\downarrow,\uparrow} = t \C{2}{\downarrow}\c{1}{\downarrow}\ket{\downarrow\uparrow,\uparrow} + U\ket{\uparrow\downarrow,\uparrow} \\
=t \ket{\uparrow,\downarrow\uparrow} + U\ket{\uparrow\downarrow,\uparrow} = -t \ket{\uparrow,\uparrow\downarrow} + U\ket{\uparrow\downarrow,\uparrow}\\
\ham \ket{\uparrow,\uparrow\downarrow} = -t \C{1}{\downarrow}\c{2}{\downarrow}\ket{\uparrow,\uparrow\downarrow} + U\ket{\uparrow,\uparrow\downarrow} = t \C{1}{\downarrow}\c{2}{\downarrow}\ket{\uparrow,\downarrow\uparrow} + U\ket{\uparrow,\uparrow\downarrow} \\
=t \ket{\downarrow\uparrow,\uparrow} + U\ket{\uparrow,\uparrow\downarrow} = -t \ket{\uparrow\downarrow,\uparrow} + U\ket{\uparrow,\uparrow\downarrow}\\
\eeq
\beq
\bordermatrix{~ & \ket{\uparrow\downarrow,\uparrow} & \ket{\uparrow,\uparrow\downarrow} \cr
              \ket{\uparrow\downarrow,\uparrow} & U & -t \cr \\
              \ket{\uparrow,\uparrow\downarrow} & -t & U \cr}
\eeq
\end{itemize}
This matrix has eigenvalues \(U \mp t\), and corresponding eigenvectors \(\frac{\ket{\uparrow\downarrow,\uparrow} \pm \ket{\uparrow,\uparrow\downarrow}}{\sqrt{2}}\)
\subsection{N=4}
With four electrons, the only possible state is \(\ket{\uparrow\downarrow,\uparrow\downarrow}\). Its easy to find the eigenvalue. Since all states are filled, no hopping can take place, so the hopping term is zero. Therefore,
\beq
\ham \ket{\uparrow\downarrow,\uparrow\downarrow} = 2U \ket{\uparrow\downarrow,\uparrow\downarrow}
\eeq
So, \(\ket{\uparrow\downarrow,\uparrow\downarrow}\) is an eigenvector with eigenvalue \(2U\).
\subsection{N=2}
This is the eigenvalue that has the largest subspace.
\begin{itemize}
\item \(S_z^{tot} = -1\): \(\ket{\downarrow,\downarrow}\)
\item \(S_z^{tot} = +1\): \(\ket{\uparrow,\uparrow}\)
\item \(S_z^{tot} = 0\):  \(\ket{\uparrow,\downarrow},\ket{\downarrow,\uparrow},\ket{0,\uparrow\downarrow},\ket{\uparrow\downarrow,0}\)
\end{itemize}

\subsubsection{\(S_z^{tot} = \pm 1\)}
These two subspaces have a single state each, so theya are obviously eigenstates. Since they both have identical spins on both sites, the hopping term is 0, and the \(U\)-term is also zero because of single occupation. As a result, they both have zero eigenvalue
\beq
\ham \ket{\downarrow,\downarrow} = \ham \ket{\uparrow,\uparrow} = 0
\eeq
\subsubsection{\(S_z^{tot} = 0\)}
This subspace has four eigenvectors,
\beq
\ket{\uparrow,\downarrow},\;\:\;\:\;\:\ket{\downarrow,\uparrow},\;\:\;\:\;\:\ket{0,\uparrow\downarrow},\;\:\;\:\;\:\ket{\uparrow\downarrow,0}
\eeq
so it is not possible to directly diagonalize this subspace. First we organize these states into eigenstates of parity. It is easy by inspection.
\beq
\hat{P}\rr{\ket{\uparrow,\downarrow}\pm\ket{\downarrow,\uparrow}} &= \pm \rr{\ket{\uparrow,\downarrow}\pm\ket{\downarrow,\uparrow}} \\
\hat{P}\rr{\ket{\uparrow\downarrow,0}\pm\ket{0,\uparrow\downarrow}} &= \pm \rr{\ket{\uparrow\downarrow,0}\pm\ket{0,\uparrow\downarrow}}
\eeq
I have the parity eigenstates for this subspace, so its most convenient to work in the basis of these eigenstates
\begin{itemize}
\item \(\hat{P} = 1: \frac{\ket{\uparrow,\downarrow}+\ket{\downarrow,\uparrow}}{\sqrt{2}},\;\:\;\:\;\:\frac{\ket{\uparrow\downarrow,0}+\ket{0,\uparrow\downarrow}}{\sqrt{2}}\)
\item \(\hat{P} = -1: \frac{\ket{\uparrow,\downarrow}-\ket{\downarrow,\uparrow}}{\sqrt{2}},\;\:\;\:\;\:\frac{\ket{\uparrow\downarrow,0}-\ket{0,\uparrow\downarrow}}{\sqrt{2}}\) 
\end{itemize}
Each eigenvalue subspace can now be diagonalized separately. First I look at the eigenstates of \(\hat{P} = 1\). I find the matrix of \(\ham\) in the subspace spanned by these two vectors and then diagonalize that subspace.
\beq
\ham\:\frac{\ket{\uparrow,\downarrow}+\ket{\downarrow,\uparrow}}{\sqrt{2}} &= -\frac{t}{\sqrt{2}} \cc{\rr{\C{1}{\downarrow}\c{2}{\downarrow}+\C{2}{\uparrow}\c{1}{\uparrow}}\ket{\uparrow,\downarrow} +\rr{\C{1}{\uparrow}\c{2}{\uparrow}+\C{2}{\downarrow}\c{1}{\downarrow}}\ket{\downarrow,\uparrow}} \\
&= -\frac{t}{\sqrt{2}}\cc{\ket{\downarrow\uparrow,0}+\ket{0,\uparrow\downarrow}+\ket{\uparrow\downarrow,0}+\ket{0,\downarrow\uparrow}} = 0 \\
\ham\:\frac{\ket{\uparrow\downarrow,0}+\ket{0,\uparrow\downarrow}}{\sqrt{2}} &= -\frac{t}{\sqrt{2}} \cc{\rr{\C{2}{\uparrow}\c{1}{\uparrow}+\C{2}{\downarrow}\c{1}{\downarrow}}\ket{\uparrow\downarrow,0} + \rr{\C{1}{\uparrow}\c{2}{\uparrow}+\C{1}{\downarrow}\c{2}{\downarrow}}\ket{0,\uparrow\downarrow}} +U\frac{\ket{\uparrow\downarrow,0}+\ket{0,\uparrow\downarrow}}{\sqrt{2}} \\
&= -\frac{t}{\sqrt{2}}\cc{\ket{\downarrow,\uparrow}-\ket{\uparrow,\downarrow}+\ket{\uparrow,\downarrow}-\ket{\downarrow,\uparrow}} +U\frac{\ket{\uparrow\downarrow,0}+\ket{0,\uparrow\downarrow}}{\sqrt{2}} = U\frac{\ket{\uparrow\downarrow,0}+\ket{0,\uparrow\downarrow}}{\sqrt{2}}
\eeq
We get the following matrix
\beq
\bordermatrix{~ & \frac{\ket{\uparrow,\downarrow}+\ket{\downarrow,\uparrow}}{\sqrt{2}} & \frac{\ket{\uparrow\downarrow,0}+\ket{0,\uparrow\downarrow}}{\sqrt{2}} \cr
              \frac{\ket{\uparrow,\downarrow}+\ket{\downarrow,\uparrow}}{\sqrt{2}} & 0 & 0 \cr \\
              \frac{\ket{\uparrow\downarrow,0}+\ket{0,\uparrow\downarrow}}{\sqrt{2}} & 0 & U \cr}
\eeq
As it appears, the subspace is already diagonal in the eigenbasis of \(\hat{P}\). The \(\hat{P} = 1\) eigenstates are eigenstates of \(\ham\), with eigenvalues 0 and \(U\). Next I look at the eigenstates of \(\hat{P}=-1\).

\beq
\ham\:\frac{\ket{\uparrow,\downarrow}-\ket{\downarrow,\uparrow}}{\sqrt{2}} &= -\frac{t}{\sqrt{2}} \cc{\rr{\C{1}{\downarrow}\c{2}{\downarrow}\C{2}{\uparrow}\c{1}{\uparrow}}\ket{\uparrow,\downarrow} -\rr{\C{1}{\uparrow}\c{2}{\uparrow}+\C{2}{\downarrow}\c{1}{\downarrow}}\ket{\downarrow,\uparrow}} \\
&= -\frac{t}{\sqrt{2}}\cc{\ket{\downarrow\uparrow,0}+\ket{0,\uparrow\downarrow}-\ket{\uparrow\downarrow,0}-\ket{0,\downarrow\uparrow}} \\
&= 2t\frac{\ket{\uparrow\downarrow,0}-\ket{0,\uparrow\downarrow}}{\sqrt{2}} \\
\ham\:\frac{\ket{\uparrow\downarrow,0}-\ket{0,\uparrow\downarrow}}{\sqrt{2}} &= -\frac{t}{\sqrt{2}} \cc{\rr{\C{2}{\uparrow}\c{1}{\uparrow}+\C{2}{\downarrow}\c{1}{\downarrow}}\ket{\uparrow\downarrow,0} - \rr{\C{1}{\uparrow}\c{2}{\uparrow}+\C{1}{\downarrow}\c{2}{\downarrow}}\ket{0,\uparrow\downarrow}} +U\frac{\ket{\uparrow\downarrow,0}+\ket{0,\uparrow\downarrow}}{\sqrt{2}} \\
&= -\frac{t}{\sqrt{2}}\cc{\ket{\downarrow,\uparrow}-\ket{\uparrow,\downarrow}-\ket{\uparrow,\downarrow}+\ket{\downarrow,\uparrow}} +U\frac{\ket{\uparrow\downarrow,0}+\ket{0,\uparrow\downarrow}}{\sqrt{2}} \\
&= 2t\frac{\ket{\uparrow,\downarrow}-\ket{\downarrow,\uparrow}}{2} + U\frac{\ket{\uparrow\downarrow,0}-\ket{0,\uparrow\downarrow}}{\sqrt{2}}
\eeq
\beq
\bordermatrix{~ & \frac{\ket{\uparrow,\downarrow}-\ket{\downarrow,\uparrow}}{\sqrt{2}} & \frac{\ket{\uparrow\downarrow,0}-\ket{0,\uparrow\downarrow}}{\sqrt{2}} \cr
              \frac{\ket{\uparrow,\downarrow}-\ket{\downarrow,\uparrow}}{\sqrt{2}} & 0 & 2t \cr \\
              \frac{\ket{\uparrow\downarrow,0}-\ket{0,\uparrow\downarrow}}{\sqrt{2}} & 2t & U \cr}
\eeq
This subspace is not automatically diagonal, but is easily diagonalized. The eigenvectors are 
\beq
&\frac{1}{N_\pm}\cc{2t \frac{(\ket{\uparrow,\downarrow}-\ket{\downarrow,\uparrow})}{\sqrt{2}} + \frac{U\pm\sqrt{U^2+16 t^2}}{2} \frac{(\ket{\uparrow\downarrow,0}-\ket{0,\uparrow\downarrow})}{\sqrt{2}}} \\
&N_{\pm} = \cc{\frac{U}{2}\qq{U\pm\sqrt{U^2+16t^2}}+16t^2}^\frac{1}{2}
\eeq

with eigenvalues \(\frac{U\pm\sqrt{U^2+16 t^2}}{2}\) respectively.
\subsection{The total spectrum}
The final spectrum is already obtained. One final thing to do is to just add the respective values of \(-\mu N\) to the eigenvalues.
\begin{table}[htb]
\begin{center}
\begin{tabular}{@{}ccccc@{}}
\toprule
\(\hat{N}\) & \(S_z^{tot}\) & \(\hat{P}\) & E & \(\ket{\Phi}\)\\
\toprule
0 & - & - & 0 & \(\ket{0,0}\) \\ \toprule
\multicolumn{1}{c}{\multirow{4}{*}{1}} & \multirow{2}{*}{-1} & 1  & -t-\(\mu\)  & \(\frac{\ket{\downarrow,0}+\ket{0,\downarrow}}{\sqrt{2}}\)  \\ \cmidrule(l){3-5} 
\multicolumn{1}{c}{}                   &                     & -1 & t-\(\mu\)   & \(\frac{\ket{\downarrow,0}-\ket{0,\downarrow}}{\sqrt{2}}\)  \\ \cmidrule(l){2-5}
\multicolumn{1}{c}{}                   & \multirow{2}{*}{1}  & 1  & -t-\(\mu\)  & \(\frac{\ket{\uparrow,0}+\ket{0,\uparrow}}{\sqrt{2}}\)  \\ \cmidrule(l){3-5} 
\multicolumn{1}{c}{}                   &                     & -1 & t-\(\mu\)   & \(\frac{\ket{\uparrow,0}-\ket{0,\uparrow}}{\sqrt{2}}\)  \\ \toprule
\multirow{6}{*}{2}                     & -1                  & 1  & 0-\(2\mu\)   & \(\ket{\downarrow,\downarrow}\)  \\ \cmidrule(l){2-5} 
                                       & \multirow{4}{*}{0}  & 1  & 0-\(2\mu\)   & \(\frac{\ket{\uparrow,\downarrow}+\ket{\downarrow,\uparrow}}{\sqrt{2}}\)  \\ \cmidrule(l){3-5} 
                                       &                     & 1  & U-\(2\mu\)   & \(\frac{\ket{\uparrow\downarrow,0}+\ket{0,\uparrow\downarrow}}{\sqrt{2}}\)  \\ \cmidrule(l){3-5} 
                                       &                     & -1 & \(\frac{U+\sqrt{U^2+16 t^2}}{2}\)-\(2\mu\)    & \(\frac{1}{N_\pm}\cc{2t \frac{(\ket{\uparrow,\downarrow}-\ket{\downarrow,\uparrow})}{\sqrt{2}} + \frac{U\pm\sqrt{U^2+16 t^2}}{2} \frac{(\ket{\uparrow\downarrow,0}-\ket{0,\uparrow\downarrow})}{\sqrt{2}}}\)  \\ \cmidrule(l){3-5} 
                                       &                     & -1 & \(\frac{U-\sqrt{U^2+16 t^2}}{2}\)-\(2\mu\)    & \(\frac{1}{N_-}\cc{2t \frac{(\ket{\uparrow,\downarrow}-\ket{\downarrow,\uparrow})}{\sqrt{2}} + \frac{U-\sqrt{U^2+16 t^2}}{2} \frac{(\ket{\uparrow\downarrow,0}-\ket{0,\uparrow\downarrow})}{\sqrt{2}}}\)  \\ \cmidrule(l){2-5} 
                                       & 1                   & 1  & 0-\(2\mu\)   & \(\ket{\uparrow,\uparrow}\) \\ \toprule
\multirow{4}{*}{3}                     & \multirow{2}{*}{-1} & 1  & U-t-\(3\mu\) & \(\frac{\ket{\uparrow\downarrow,\downarrow}+\ket{\downarrow,\uparrow\downarrow}}{\sqrt{2}}\) \\ \cmidrule(l){3-5} 
                                       &                     & -1 & U+t-\(3\mu\) & \(\frac{\ket{\uparrow\downarrow,\downarrow}-\ket{\downarrow,\uparrow\downarrow}}{\sqrt{2}}\) \\ \cmidrule(l){2-5}
                                       & \multirow{2}{*}{1}  & 1  & U-t-\(3\mu\) & \(\frac{\ket{\uparrow\downarrow,\uparrow}+\ket{\uparrow,\uparrow\downarrow}}{\sqrt{2}}\) \\ \cmidrule(l){3-5} 
                                       &                     & -1 & U+t-\(3\mu\) & \(\frac{\ket{\uparrow\downarrow,\uparrow}-\ket{\uparrow,\uparrow\downarrow}}{\sqrt{2}}\) \\ \toprule
4                                      & 0                   & 1  & 2U-\(4\mu\)  & \(\ket{\uparrow\downarrow,\uparrow\downarrow}\) \\
\toprule
\end{tabular}
\end{center}
\end{table}
\end{document}

